
\documentclass[UTF8, a4paper, 11pt, onecolumn]{article}
% save as tex UTF8 coding, run as pdfTexify
\usepackage{CTEX}

\usepackage{amsmath,amsfonts,amssymb} % Math packages
\usepackage{bm} %%\bm{}来加粗

\def \Pr{\mathrm{Pr}}
\def \exp{\textrm{exp}}


\begin{document}



\begin{center}
{\Large {\bf Standard deviation and standard error(2018/05/12)}} \\
\end{center}
标准差与标准误
Standard deviation and standard error
(Standard deviation and standard error)


standard error 是抽样分布中样本平均数的标准差。因为“抽样分布”这个分布中,所存在的个体数并不是$n$。
$n$是什么呢?$n$是我们进行随机抽样的时候每一个样本的sample size。

两个概念的联系与区别

联系:
二者都是标准差。

区别: 

标准差(standard deviation),某个随机变量的标准差,衡量的是该随机变量的离散度。

标准误(standard error),也叫抽样标准误,是样本统计量的标准差,衡量的是抽样分布的离散度,对应的随机变量是样本统计量。比如样本均值的标准误(standard error of sample mean),衡量的就是样本均值的离散度。


一般情况下,你不知道一个样本底下的随机变量是什么,但是你一般可以通过样本的各阶矩来估计该随机变量的各阶矩(总体矩)。由于统计量的分布依赖于总体的分布,统计量的矩也依赖于总体的矩。你拿用样本算出来的(总体矩的)估计量,去计算统计量的矩的估计量,就得到了抽样分布的矩。



\end{document}
